% .:: Laden der LaTeX4EI Formelsammlungsvorlage
\documentclass[fs, footer]{latex4ei}
\usepackage[european]{circuitikz}

\usepackage{multirow}
\usepackage{latexnew}



% Dokumentbeginn
% ======================================================================
\begin{document}


% Aufteilung in Spalten
\vspace{-4mm}
\begin{multicols*}{4}
	\fstitle{Systemtheorie}

	\emphbox{
	\textbf{Wichtiger Hinweis}
	\\ Diese Formelsammlung ist noch in der Entwicklung und nicht prüfungstauglich ! \\ Allerdings würden wir uns über Unterstützung freuen das zu ändern. Wer Lust hat kann uns über das Kontaktformular auf www.latex4ei.de erreichen.
	}
\section{Reaktive Elemente}
% ===============================================================================================

	\subsection{Die vier zentralen Größen $u,i,q,\Phi$}
	% ----------------------------------------------------------------------
	... beschreiben die Wirkungsweise von elektronischen Bauelementen.\\ \\
	\textbf{Spannung \textit{u}}: Potentialdifferenz. Hohes zu niedrigem Potential\\
	\textbf{Strom \textit{i}}: Bewegte Ladung. Bewegungsrichtung positiver Ladung\\
	\textbf{Ladung \textit{q}}: Grundeigenschaft von Materie.\\
	\textbf{Magnetischer Fluss \textit{$\Phi$}}: Grundeigenschaft von elektr. magn. Feldern\\
		\subsubsection{Allgemeine Zusammenhänge $u,i,q,\Phi$}
		Ladung und Strom beschreiben den Zustand der Materie.\\
		Spannung und magn. Fluss beschreiben den Zustand des elekt. magn. Feldes.\\
		Kondensator ist u-gesteuert (q-gesteuert), falls für ein u (q) nur ein q (u)  existiert. \\
		Induktivität ist i-gesteuert ($\phi$-gesteuert), falls für ein i ($\phi$) nur ein $\phi$ (i) existiert. \\
		\begin{tabular}{l|l}
			$i(t) = \dot q(t)$ & $[i]=A$\\
			$q(t) = q(t_0) + \int_{t_0}^t i(\tau) \mathrm d\tau$	& $[q]=As=C$ \\ \hline
			$u(t) = \dot \Phi(t)$ & $[u]=V$\\
			$\Phi = \Phi(t_0) + \int_{t_0}^t u(\tau) \mathrm d\tau$ & $[\Phi]=Vs=Wb$ \\
		\end{tabular}




		\subsubsection{Arten von Bauelementen}
		\begin{tabular}{l|l|l|l}
			Art & Symbol & Beschr. & linear\\ \hline
			Resistivität & \includegraphics[height=0.4cm]{./img/Resistivitat.pdf} & $f_R(u,i)$  & $u = U_0 + R \cdot i$\\
			Kapazität & \includegraphics[height=0.4cm]{./img/Kapazivitat.pdf} & $f_C(u,q)$ & $q = Q_0 + C \cdot u$\\
			Induktivität & \includegraphics[height=0.4cm]{./img/Induktivitat.pdf} & $f_L(i,\Phi)$ & $\Phi = \Phi_0 + L \cdot i$\\
			Memristivität & \includegraphics[height=0.4cm]{./img/Memristivitat.pdf} & $f_M(q,\Phi)$ & $\Phi = \Phi_0 + M \cdot q$\\
		\end{tabular}
		\subsubsection{Zusammenhang der Bauelemente}
		\begin{center}
			\includegraphics[scale=0.3]{./img/reactance_overview.pdf}
		\end{center}
		\subsubsection{Eigenschaften von Reaktanzen}
		\textbf{Linearität}: siehe Eintore\\
		\textbf{Differentialgleichung}: $i(t) = C \frac{\mathrm du(t)}{\mathrm dt}, u(t) = L \frac{\mathrm di(t)}{\mathrm dt}$\\
		\textbf{Gedächtnis}: Verhalten durch vorhergehende Klemmengrößen bestimmt.\\
		\textbf{Stetigkeit}: $u_C(t)$, $i_L(t)$ stetig in $(t_a, t_b)$, wenn Torgrößen endlich\\
		\textbf{Verlustfreiheit}: $W_C(t_1, t_2) = \int_{t_1}^{t_2}\! u(t)i(t)\,\mathrm dt = \int_{q_1}^{q_2}\! \mathrm{X}(q)\,\mathrm{d}t$ (Arbeit)\\
		Falls linear: $W = \frac{Cu^2}{2} = \frac{Li^2}{2}$\\
		Periodisch: $u(t+T) = u(t)$, $q(t+T) = q(t)$\\
		Graphisch: Falls keine geschlossenen Schleifen in q/u, $\Phi$/i-Diagramm existiert (Hystenesefrei)\\
		\textbf{Energie (nicht linearer Fall)}:\\
		- Kapazitiv: $W_C(t_1, t_2) = \int_{t_1}^{t_2} \! u(t)i(t)\, \mathrm dt = \int_{q_1}^{q_2} \! u(q)\, \mathrm dq$\\
		- Induktiv: $W_C(t_1, t_2) = \int_{t_1}^{t_2} \! u(t)i(t)\, \mathrm dt = \int_{\Phi_1}^{\Phi_2} \! i(\Phi)\, \mathrm d\Phi$\\
		\textbf{Energie (linearer Fall)}:\\
		- Kapazitiv: $W_C = \frac{C}{2}u^2 = \frac{1}{2C}q^2$\\
		- Induktiv: $W_L = \frac{L}{2}i^2 = \frac{1}{2L} \Phi^2$\\
		Graphisch: Fläche zwischen der Kennlinie und der q-/$\Phi$-Achse\\
		\textbf{Relaxationspunkte (=Ruhepunkte)}: Betriebspunkte, in dem die in einer Reaktanz gespeicherte Energie minimal ist. Kandidaten sind: Extremwerte, Wendepunkte, Knicke, Schnittpunkte mit q-/$\Phi$-Achse\\
		\subsubsection{Verschaltung von Reaktanzen}
		- Parallelschaltung: $C_p = C_1 + C_2$, $L_p = L_1 || L_2 = \frac{L_1L_2}{L_1+L_2}$\\
		- Serienschaltung: $C_p = C_1 || C_2 = \frac{C_1C_2}{C_1+C_2}$, $L_p = L_1 + L_2$\\


Merke: Am Kondensa\textsl{tor}, eilt der Strom \textsl{vor}, bei Induktivi\textsl{täten}, wird er sich ver\textsl{späten}\\
Merke: Ist das Mädchen brav, bleibt der Bauch konkav, hat das Mädchen Sex, wird der Bauch konvex.\\
% Liste mit Eselsbrücken für Ingenieure
% selbstausdenken begriffspaare
\section{Schaltungen ersten Grades}
\sectionbox{
	\textbf{I. Resistives ESB bestimmen}\\
\tablebox{
	\begin{tabular*}{\columnwidth}{@{\extracolsep\fill}l|ll@{}} \trule
	& Kapazität & Induktivität \\ \mrule
	ESB-Typ & Helmholtz-Thévinin & Mayer-Norton\\
	%TODO Bild einfügen :)
	Zustandsgröße & $x(t) = u_C(t)$ & $x(t) = i_L(t)$\\
	Zeitkonstante & $\tau = RC$ & $\tau = GL$\\
		\brule
	\end{tabular*}
}
	
	\textbf{II. Aufstellen DGL:} $\dot x(t) = - \fr{1}{\tau}x(t) + \fr{1}{\tau}v$ mit der Erregung $v$\\
	\textbf{III. Lösen der DGL:}\\
	Konstante Erregung:  $x(t) = x_\infty + (x_0 - x_\infty )\e^{-\frac{t-t_0}{\tau}}$ \\
	Allgemeine Erregung: $x(t) = x_0\e^{\frac{t_0-t}{\tau}}+\int_{t_0}^t{\frac{1}{\tau}v(t')\e^{\frac{t'_0-t'}{\tau}}} $ \\
	zero-input-response: Erster Summand\\
	zero-state-response: Zweiter Summand\\
	\textbf{IV. Dynamischer Pfad}\\
\tablebox{
	\begin{tabular*}{\columnwidth}{@{\extracolsep\fill}ll@{}} \trule
	Kapazität & Induktivität\\ \mrule
	$i < 0$: $u$ wird größer & $u < 0$: $i$ wird größer\\
	$i > 0$: $u$ wird kleiner & $u > 0$: $i$ wird kleiner\\
	$i = 0$: GGP & $u = 0$: GGP\\ 	
	\brule
	\end{tabular*}
}
\textbf{Toter Punkt:} kein GGP, aber Pfad kann nicht fortgesetzt werden \ra Sprung der nicht stetigen Größe ($i_C$ oder $u_L$)\\
\textbf{Gleichgewichtspunkt (GGP):}
\begin{itemize}
	\item[a)] stabil, falls der Pfad nicht aus diesem Punkt herausläuft
	\item[b)] instabil, falls der Pfad aus dem Punkt herausläuft
	\item[c)] virtuell, falls der Pfad in einen toten Punkt läuft auf dem verlängertem Pfad auf der Achse
\end{itemize}
}

\subsection{Stabile Schaltung ($\tau > 0$)}
%TODO Bilder von Graph
\subsection{Instabile Schaltung ($\tau < 0$)}
%TODO Bilder von Graph


\end{multicols*}
\end{document}

